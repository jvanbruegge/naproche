\documentclass{article}

\usepackage[utf8]{inputenc}
\usepackage[english]{babel}
\usepackage{../lib/tex/naproche}

\title{Morse Kelley set theory}
\author{}
\date{}

\begin{document}
  \pagenumbering{gobble}

  \maketitle

  This module implements the Morse-Kelley (MK) set theory,
  an extension of ZFC. Any lemma that can be proven in ZFC
  also can be proven in MK, but unlike ZFC, MK is finitely axiomatizable,
  that is, we can describe it without using schemata over arbitrary formulas.
  The axiom schema of class formation has been omitted in this presentation
  because it is built into the compiler.

  \begin{forthel}
    [synonym relation/-s]
    [synonym function/-s]
    [synonym element/-s]

    Let $B, C, D, E$ denote classes.
    Let $b, c, d, e$ denote sets.
    Let $w, x, y, z$ denote objects.

    \begin{axiom}
      $b$ is a class.
    \end{axiom}
    
    \begin{axiom}
      $b$ is an object.
    \end{axiom}

    \begin{definition}
      $x$ is in $B$ iff $x$ is an element of $B$.
    \end{definition}

    \begin{definition}
      $x \in B$ iff $x$ is in $B$.
    \end{definition}

    \begin{definition}
      $x \notin B$ iff $x$ is not in $B$.
    \end{definition}

    %This axiom is implicitly used in the compiler.
    \begin{axiom}[Ext]
      If for every object $x$ $x$ is an element of $C$ iff $x$ is an element of $B$ then $C = B$.
    \end{axiom}

    \begin{signature}
      The empty set is a set.
    \end{signature}

    \begin{axiom}[Empty]
      $x$ is not an element of the empty set.
    \end{axiom}

    \begin{lemma}
      The empty set is $\{ u | u \neq u \}$.
    \end{lemma}

    \begin{definition}
      B is nonempty iff there is a x such that x is in B.
    \end{definition}

    \begin{lemma}
      If B is nonempty then B is not the empty set.
    \end{lemma}

    \begin{definition}
      B is empty iff B is not nonempty.
    \end{definition}

    \begin{lemma}
      B is empty iff B is { u | u != u }.
    \end{lemma}

    \begin{signature}
      The pair of x and y is a set.
    \end{signature}

    \begin{axiom}[Pair]
      x is in the pair of y and z iff x = y or x = z.
    \end{axiom}

    \begin{lemma}
      The pair of x and y is { u | u = x or u = y }.
    \end{lemma}

    \begin{definition}
      The singleton of x is the pair of x and x.
    \end{definition}

    \begin{definition}
      The ordered pair of x and y is the pair of x and the pair of x and y.
    \end{definition}

    \begin{definition}
      (x, y) is the ordered pair of x and y.
    \end{definition}

    \begin{lemma}[OrdPair]
      If the ordered pair of x and y is equal to the ordered pair of z and w
        then x = z and y = w.
    \end{lemma}

    \begin{definition}
      The union of D is a class B such that (for every object $x$ x is in B iff there is a c 
        such that c is in D and x is in c).
    \end{definition}

    \begin{axiom}[Union]
      The union of d is a set.
    \end{axiom}

    \begin{definition}
      The union of D and B is a class C such that for every object $x$
        $x$ is in C iff $x$ is in D or $x$ is in $B$.
    \end{definition}

    \begin{definition}
      \Union{D, B} is the union of D and B.
    \end{definition}

    \begin{lemma}
      \Union{d, b} is the union of the pair of d and b.
    \end{lemma}

    \begin{lemma}
      \Union{d, b} is a set.
    \end{lemma}

    \begin{definition}
      A subclass of B is a class C such that
        for all x if x is an element of C then x is an element of B.
    \end{definition}

    \begin{definition}
      A subset of b is a set c 
        such that c is a subclass of b.
    \end{definition}

    \begin{signature}
      The powerset of b is a set.
    \end{signature}

    \begin{axiom}[PowerSet]
      b is the powerset of d iff (for every set c c is in b iff c is a subclass of d).
    \end{axiom}

    \begin{definition}
      The successor of b is the pair of b and the singleton of b.
    \end{definition}

    \begin{axiom}[Inf]
      There is a set n such that the empty set is in n and for every b
        such that b is in n the successor of b is in n.
    \end{axiom}

    \begin{definition}
      The product of C and B is { (u, v) | u is a object and v is a object and u is in C and v is in B }.
    \end{definition}

    \begin{signature}
      A relation is a class.
    \end{signature}

    \begin{axiom}[RelationIntro]
      C is a relation iff for every x such that x is in C
        there is a y such that there is a z such that x is the ordered pair of y and z.
    \end{axiom}

    Let R, S denote relations.

    \begin{definition} 
      The domain of R is { u | there is a object v such that (u, v) is in R }.
    \end{definition}

    \begin{definition} 
      The range of R is { v | there is a object u such that (u, v) is in R }.
    \end{definition}

    \begin{definition} 
      The restriction of R to B is { (u, v) | u is a object and v is a object and (u, v) is in R and u is in B }.
    \end{definition}

    \begin{definition} 
      The image of B under R is { v | there is a object u such that u is in B and (u, v) is in R }.
    \end{definition}

    \begin{definition} 
      The preimage of B under R is { u | there is a object v such that v is in B and (u, v) is in R }.
    \end{definition}

    \begin{definition} 
      The composition of S and R is { (x, z) | x is a object and z is a set and there is a object y such that (x, y) is in R and (y, z) is in S }.
    \end{definition}

    \begin{definition} 
      The inverse of R is { (y, x) | x is a object and y is a object and (x, y) is in R }.
    \end{definition}

    \begin{definition} 
      \Dom{R} is the domain of R.
    \end{definition}

    \begin{definition} 
      \Ran{R} is the range of R.
    \end{definition}

    \begin{definition}
      The field of R is a class C such that C is \Union{\Dom{R}, \Ran{R}}.
    \end{definition}

    \begin{definition} 
      \Field{R} is the field of R.
    \end{definition}

    \begin{definition} 
      R[B] is the image of B under R.
    \end{definition}

    \begin{definition} 
      R^{-1} is the inverse of R.
    \end{definition}

    \begin{definition}
      R is reflexive iff for all x such that x is in the field of R (x, x) is in R.
    \end{definition}

    \begin{definition}
      R is irreflexive iff for all x such that x is in the field of R (x, x) is not in R.
    \end{definition}

    \begin{definition}
      R is symmetric iff for all x, y such that (x, y) is in R (y, x) is in R.
    \end{definition}

    \begin{definition}
      R is antisymmetric iff for all x, y such that (x, y) is in R and (y, x) is in R x = y.
    \end{definition}

    \begin{definition}
      R is transitive iff for all x, y, z such that (x, y) is in R and (y, z) is in R (x, z) is in R.
    \end{definition}

    \begin{definition}
      R is connex iff for all x, y such that x,y are in R
        (x, y) is in R or (y, x) is in R or x = y.
    \end{definition}

    \begin{signature}
      A function is a relation.
    \end{signature}

    \begin{axiom}[FunctionIntro]
      R is a function iff for all x, y, z such that (x, y) is in R and (x, z) is in R y = z.
    \end{axiom}

    Let $F, G$ denote functions.

    \begin{definition}
      Assume \Dom{F} is not empty.
      The value of F at x is a object y such that (x, y) is in F.
    \end{definition}

    \begin{definition}
      F(x) is the value of F at x.
    \end{definition}

    \begin{axiom}[Choice]
      Let c be a set such that for all b such that b is in c b is nonempty.
        There exists a function F such that \Dom{F} = c
        and for all x such that x is in c there is a set d such that x is in d and $d = F(x)$.
    \end{axiom}

    \begin{axiom}[Replacement]
      The restriction of F to c is a set.
    \end{axiom}

    \begin{definition}
      The intersection of B is { v | for every set u such that u is in B v is in u }.
    \end{definition}

    \begin{definition}
      The intersection of c and b is the intersection of the pair of c and b.
    \end{definition}

    \begin{axiom}[Restriction]
      If c is nonempty then there is a b such that b is in c and
        (the intersection of c and b) is empty.
    \end{axiom}
  \end{forthel}
\end{document}