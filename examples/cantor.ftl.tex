\documentclass{article}

\usepackage[utf8]{inputenc}
\usepackage[english]{babel}
\usepackage{../lib/tex/naproche}

\title{Cantor's theorem}
\author{}
\date{}

\begin{document}
  \pagenumbering{gobble}

  \maketitle

  In this document we prove Cantor's theorem:

  \begin{quotedtheorem}
    There is no surjection defined on a set $M$ that surjects onto the powerset of $M$.
  \end{quotedtheorem}

  \begin{forthel}
    [synonym surject/-s]
    [readtex MorseKelley.ftl.tex]

    Let $M$ denote a set. Let $f$ denote a function.
    Let $f$ is defined on $M$ stand for $\Dom{f} = M$.

    \begin{definition}
      $f$ surjects onto $M$ iff
        for every object $x$ if $x$ is in $M$ then there is an object $y$ such that
          $x$ is equal to the value of $f$ at $y$ and $y$ is in the domain of $f$.
    \end{definition}

    \begin{lemma}
      If $f$ surjects onto the powerset of $M$ then for every object $z$ there is a set $S$
        such that $S = f(z)$.
    \end{lemma}

    \begin{theorem}[Cantor]
      No function $f$ that is defined on $M$ surjects onto the powerset of $M$.
    \end{theorem}
    \begin{proof}
      Assume the contrary.
      Take a set $N$ such that $N = \{ x "in" M \mid "there is a set S such that S = f(x) and" x \notin S \}$.
      $N$ is a subset of $M$.
      Take an object $z$ such that $z$ is in $M$ and $f(z) = N$.
      Case $z \in N$.
        Take a set $S$ such that $S = f(z)$ and $z \notin S$.
        Then $S = f(z) = N$. Then $z \notin N$.
      end.
      Case $z \notin N$.
        Take a set $S$ such that $S = f(z)$.
        Case $z \notin S$.
          $z \in N$.
        end.
        Case $z \in S$.
          Then $z \in N$.
        end.
      end.
    \end{proof}
  \end{forthel}
\end{document}
