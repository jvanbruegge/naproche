\documentclass{article}

\usepackage[utf8]{inputenc}
\usepackage[english]{babel}
\usepackage{tex/naproche}

\title{Prelude}
\author{}
\date{}

\begin{document}
  \pagenumbering{gobble}

  \maketitle

  This module implements a "Prelude" for Naproche: Those notions that were built-in
  in the last release and need to be specified in Naproche texts now.
  It is a shortened version of the Morse Kelley file in the same directory,
  since the longer version adds many useful notions that make it harder
  for provers to prove the tasks that they could prove before.

  \begin{forthel}
    [synonym relation/-s]
    [synonym function/-s]

    Let $B, C, D$ denote classes.
    Let $s, t, u$ denote sets.
    Let $w, x, y, z$ denote objects.

    \begin{axiom}[SubclassOfSet]
      Assume for any object x such that x is an element of $B$ x is an element of $s$. Then $B$ is a set.
    \end{axiom}

    \begin{definition}
      $x \in B$ iff $x$ is an element of $B$.
    \end{definition}

    \begin{definition}
      $x \notin B$ iff not $x \in B$.
    \end{definition}

    %This axiom is implicitly used in the compiler.
    \begin{axiom}[Ext]
      If for every object $x$ $x$ is an element of $C$ iff $x$ is an element of $B$ then $C = B$.
    \end{axiom}

    \begin{signature}
      (x, y) is an object.
    \end{signature}

    \begin{axiom}[OrdPair]
      If (x, y) = (z, w) then x = z and y = w.
    \end{axiom}

    \begin{definition}
      The product of C and B is { (u, v) | u is an element of C and v is an element of B }.
    \end{definition}

    \begin{signature}
      A function is a class.
    \end{signature}

    \begin{axiom}[RelationIntro]
      $C$ is a function iff for every element $x$ of $C$
          there is a $y$ such that there is a $z$ such that $x = (y, z)$
          and for all $x, y, z$ such that $(x, y) \in C$ and $(x, z) \in C$ $y = z$.
    \end{axiom}

    Let $F, G$ denote functions.

    \begin{definition} 
      \Dom{F} = { object u | there is a object v such that $(u, v) \in F$ }.
    \end{definition}

    \begin{signature}
      F(x) is an object.
    \end{signature}

    \begin{axiom}
      Assume $x \in \Dom{F}$.
      F(x) is a object y such that $(x, y) \in F$.
    \end{axiom}
  \end{forthel}
\end{document}
