\documentclass{article}

\usepackage[utf8]{inputenc}
\usepackage[english]{babel}
\usepackage{../tex/naproche}

\title{Maximum modulus principle}
\author{Steffen Frerix (2018), Adrian De Lon (typesetting, 2021)}
\date{}

\newcommand{\Ball}[2]{B_{#1}(#2)}
\newcommand{\image}[2]{#1^{\to}[#2]}

\begin{document}
  \pagenumbering{gobble}

  \maketitle


  \section{Preliminaries}

  \begin{forthel}
    [readtex prelude.ftl.tex]
    [synonym number/-s]

    Let $f$ denote a function.
    Let $M$ denote a class.
    Let $z$ denote an object.

    Let the domain of $f$ stand for $\Dom{f}$.
    Let $z$ is in $M$ stand for $z$ is an element of $M$.
    Let $M$ contains $z$ stand for $z$ is in $M$.

    \begin{definition}
      A subclass of $M$ is a class $N$ such that
      every element of $N$ is an element of $M$.
    \end{definition}

    \begin{definition}
      Let $f$ be a function.
      Assume $M$ is a subclass of the domain of $f$.
      $\image{f}{M} = \{ f(x) | x\in M \}$.
    \end{definition}
  \end{forthel}

    \section{Real and complex numbers}

  \begin{forthel}
    \begin{signature}
      A complex number is an object.
    \end{signature}

    Let $z, w$ denote complex numbers.

    \begin{signature}
      A complex function is a function.
    \end{signature}

    Let $f$ denote a complex function.

    \begin{signature}
      $f(z)$ is a complex number.
    \end{signature}

    \begin{axiom}
      Every element of $\Dom{f}$ is a complex number and for every element $z$ of $\Dom{f}$ $f(z)$ is a complex number.
    \end{axiom}

    \begin{axiom}
      Every element of $M$ is a complex number.
    \end{axiom}

    \begin{signature}
      A real number is an complex number.
    \end{signature}

    Let $x,y$ denote real numbers.

    \begin{signature}
      $|z|$ is a real number.
    \end{signature}

    \begin{signature}
      $x$ is positive is an atom.
    \end{signature}

    Let $\varepsilon, \delta$ denote positive real numbers.

    \begin{signature}
      $x < y$ is an atom.
    \end{signature}

    Let $x \leq y$ stand for $x = y$ or $x < y$.

    \begin{axiom}
      If $x < y$ then not $y < x$.
    \end{axiom}
  \end{forthel}

  \section{Properties of functions and open balls}

  \begin{forthel}
    \begin{signature}
      $f$ is holomorphic is an atom.
    \end{signature}

    \begin{signature}
      $\Ball{\varepsilon}{z}$ is a set.
    \end{signature}

    \begin{axiom}
       $\Ball{\varepsilon}{z}$ contains $z$.
    \end{axiom}

    \begin{axiom}
      Assume $x$ is element of $\Ball{\varepsilon}{z}$. Then $x$ is a complex number.
    \end{axiom}

    \begin{definition}
      Let $A$ be a class.
      A member of $A$ is a complex number $x$ such that $x$ is an element of $A$.
    \end{definition}

    \begin{axiom}
      $|z| < |w|$ for some member $w$ of $\Ball{\varepsilon}{z}$.
    \end{axiom}

    \begin{definition}
      $M$ is open iff for every member $z$ of $M$ there exists $\varepsilon$ such that
        $\Ball{\varepsilon}{z}$ is a subclass of $M$.
    \end{definition}

    \begin{axiom}
      $\Ball{\varepsilon}{z}$ is open.
    \end{axiom}

    \begin{definition}
      A local maximal point of $f$ is a member $z$ of the domain of $f$ such that there exists $\varepsilon$
        such that $\Ball{\varepsilon}{z}$ is a subclass of the domain of $f$ and $|f(w)| \leq |f(z)|$ for every member $w$ of $\Ball{\varepsilon}{z}$.
    \end{definition}

    \begin{definition}
      Let $U$ be a subclass of the domain of $f$.
      $f$ is constant on $U$ iff there exists an object $z$ such that $f(w) = z$ for every member $w$ of $U$.
    \end{definition}

    Let $f$ is constant stand for $f$ is constant on the domain of $f$.

    \begin{axiom}
      Assume $f$ is holomorphic and $\Ball{\varepsilon}{z}$ is a subclass of the domain of $f$.
      If $f$ is not constant on $\Ball{\varepsilon}{z}$
        then $\image{f}{\Ball{\varepsilon}{z}}$ is open.
    \end{axiom}
  \end{forthel}

    \section{Maximum modulus principle}

  \begin{forthel}
    \begin{signature}
      A region is an open set.
    \end{signature}

    \begin{axiom}[Identity theorem]
      Assume $f$ is holomorphic and the domain of $f$ is a region.
      Assume that $\Ball{\varepsilon}{z}$ is a subclass of the domain of $f$.
      If $f$ is constant on $\Ball{\varepsilon}{z}$ then $f$ is constant.
    \end{axiom}

    \begin{proposition}[Maximum modulus principle]
      Assume $f$ is holomorphic and the domain of $f$ is a region.
      If $f$ has a local maximal point then $f$ is constant.
    \end{proposition}
    \begin{proof}
      Let $z$ be a local maximal point of $f$.
      Take $\varepsilon$ such that
        $\Ball{\varepsilon}{z}$ is a subclass of $\Dom{f}$
        and $|f(w)| \leq |f(z)|$ for every element $w$ of $\Ball{\varepsilon}{z}$.

      Let us show that $f$ is constant on $\Ball{\varepsilon}{z}$.
      Proof by contradiction.
        Assume the contrary.
        Then $\image{f}{\Ball{\varepsilon}{z}}$ is open.
        We can take $\delta$ such that
          $\Ball{\delta}{f(z)}$ is a subclass of $\image{f}{\Ball{\varepsilon}{z}}$.
        Therefore there exists an element $w$ of $\Ball{\varepsilon}{z}$ such that
          $|f(z)| < |f(w)|$. Contradiction.
    	End.

      Hence $f$ is constant.
    \end{proof}
  \end{forthel}

\end{document}
