\documentclass[11pt]{article}
\usepackage{amssymb}
\usepackage{../tex/naproche}

\author{Andrei Paskevich and Peter Koepke\\
   U Paris-Saclay and U Bonn}

\title{Euclid's Proof 
of the Infinitude of Primes, Formalized in \Naproche{}}

\begin{document}

\newcommand{\val}[2]{#1_{#2}}
\newcommand{\Prod}[3]{#1_{#2} \cdots #1_{#3}}
\newcommand{\Seq}[2]{\{#1,\dots,#2\}}
\newcommand{\Set}[3]{\{#1_{#2},\dots,#1_{#3}\}}
\newcommand{\Primes}{\mathbb{P}}
\newcommand{\Naproche}{$\mathbb{N}$aproche}

\maketitle

This paper contains a standard proof of Euclid's theorem 
that there are infinitely many prime numbers. We
follow the first proof given in \emph{Proofs from THE BOOK}
by Martin Aigner and Günter M. Ziegler, Springer Verlag.
Before the proof we 
set up a language and axioms for natural number arithmetic,
define divisibility and prime numbers,
introduce some set theoretic background and define
finite sets, sequences and products.

On an older laptop 
with an Intel Pentium N3710 processor \Naproche{} takes 
around 75 seconds to check this text.


\section{\Naproche{} Settings and Notation}

\begin{forthel}

%\begin{lemma} Contradiction. \end{lemma}

[printprover on][dump on]

[readtex prelude.ftl.tex]
[synonym number/-s]
[synonym divide/-s]
[synonym set/-s] [synonym element/-s] 
[synonym belong/-s] [synonym subset/-s]

\end{forthel}


\section{Natural Numbers}
 
\begin{forthel}

\begin{signature}
  A natural number is an object.
\end{signature}

Let $i,k,l,m,n,p,q,r$ denote natural numbers.

\begin{signature} $0$ is a natural number.
\end{signature}

Let $x$ is nonzero stand for $x \neq 0$.

\begin{signature}
  $1$ is a nonzero natural number.
\end{signature}

\begin{signature} $m + n$ is a natural number.
\end{signature}

\begin{signature} $m * n$ is a natural number.
\end{signature}

\begin{axiom} $m + n = n + m$.
\end{axiom}

\begin{axiom} $(m + n) + l = m + (n + l)$.
\end{axiom}

\begin{axiom}  $m + 0 = m = 0 + m$.
\end{axiom}

\begin{axiom} $m * n = n * m$.
\end{axiom}

\begin{axiom} $(m * n) * l = m * (n * l)$.
\end{axiom}

\begin{axiom} $m * 1 = m = 1 * m$.
\end{axiom}

\begin{axiom} $m * 0 = 0 = 0 * m$.
\end{axiom}

\begin{axiom} $m * (n + l) = (m * n) + (m * l)$ and
                $(n + l) * m = (n * m) + (l * m)$.
\end{axiom}

\begin{axiom} If $l + m = l + n$ or $m + l = n + l$ 
then $m = n$.
\end{axiom}

\begin{axiom} Assume that $l$ is nonzero.
If $l * m = l * n$ or $m * l = n * l$ then $m = n$.
\end{axiom}

\begin{axiom} If $m + n = 0$ then $m = 0$ and $n = 0$.
\end{axiom}

\end{forthel}

\section{The Natural Order}

\begin{forthel}

\begin{definition} $m \leq n$ iff 
there exists a natural number $l$ such that  
$m + l = n$.
\end{definition}

Let $m < n$ stand for $m \leq n$ and $m \neq n$.

\begin{signature}
  Assume that $n \leq m$.
  $m - n$ is a natural number.
\end{signature}

\begin{axiom}
  Assume that $n \leq m$.
  $n + (m - n) = m$.
\end{axiom}

\end{forthel}

The following three lemmas show that $\leq$ is a partial order:

\begin{forthel}

\begin{lemma} $m \leq m$.
\end{lemma}

\begin{lemma} If $m \leq n \leq m$ 
then $m = n$.
\end{lemma}

\begin{lemma} If $m \leq n \leq l$ 
then  $m \leq l$.
\end{lemma}

\begin{axiom} $m \leq n$ or $n < m$.
\end{axiom}

\begin{lemma} Assume that $l < n$.
  Then $m + l < m + n$ and $l + m < n + m$.
\end{lemma}

\begin{lemma} Assume that $m$ is nonzero and $l < n$.
  Then $m * l < m * n$ and $l * m < n * m$.
\end{lemma}

\begin{axiom} $n = 0$ or $n = 1$ or $1 < n$.
\end{axiom}

\begin{lemma} If $m \neq 0$ then $n \leq n * m$.
\end{lemma}
\end{forthel}

\section{Induction}

In \Naproche{} induction is defined over classes:

\begin{forthel}
\begin{axiom}[Induction]
  Let $C$ be a class.
  If for every natural number $m$ such that (\forall n $n < m$ => $n \in C$) $m \in C$
    then every natural number is an element of $C$.
\end{axiom}
\end{forthel}

\section{Division}

\begin{forthel}

\begin{definition}
  $n$ divides $m$ iff for some $l$ $m = n * l$.
\end{definition}

Let $n | m$ denote $n$ divides $m$.
Let a divisor of $n$ denote a natural number 
that divides $n$.

\begin{lemma}
  Let $l | m$ and $l | m + n$. Then $l | n$.
\end{lemma}

\begin{proof}
Assume that $l$ is nonzero.
Take $p$ such that $m = l * p$. 
Take $q$ such that $m + n = l * q$.

Let us show that 
$p \leq q$.
Proof by contradiction.
Assume the contrary. Then $q < p$.
$m+n = l * q < l * p = m$.
Contradiction. qed.

Take $r$ such that $r = q - p$.
We have $(l * p) + (l * r) = l * q = m + n = (l * p) + n$.
Hence $n = l * r$.
\end{proof}

\begin{lemma} Let $m | n \neq 0$. Then $m \leq n$.
\end{lemma}

\end{forthel}

\section{Primes}

\begin{forthel}

Let $x$ is nontrivial stand for $x \neq 0$ and $x \neq 1$.

\begin{definition}
$n$ is prime iff $n$ is nontrivial and
    for every divisor $m$ of $n$ $m = 1$ or $m = n$.
\end{definition}

\begin{lemma} Every nontrivial $k$ has a prime divisor.
\end{lemma}

\end{forthel}

\section{Classes}

\begin{forthel}

Let $S,T$ stand for classes.
Let $x$ belongs to $S$ stand 
for $x$ is an element of $S$.

\begin{definition} A subclass of $S$ is a class $T$ 
such that every element of $T$ belongs to $S$.
\end{definition}

Let $T \subseteq S$ stand for $T$ is a subclass of $S$.
\end{forthel}

We extend the built-in ontology of \Naproche{}
according to the following principles for elements and sets:
\begin{forthel}

\begin{definition} $\mathbb{N}$ is the 
class of natural numbers.
\end{definition}

\end{forthel}

\section{Finite Sequences and Products}

\begin{forthel}
\begin{definition}
  $\Seq{m}{n}$ is { natural number i | $ m \leq i \leq n $ }.
\end{definition}

\begin{axiom} $\Seq{m}{n}$ is a set. \end{axiom}

\begin{definition} A sequence of length $n$ is a
function $F$ such that $\Dom{F} = \Seq{1}{n}$.
\end{definition}

Let $F$ stand for a function.
Let $\val{F}{i}$ stand for $F(i)$.

\begin{definition} Let $F$ be a sequence of length $n$.
$\Set{F}{1}{n} = \{ \val{F}{i} | i \in \Dom{F}\}$.
\end{definition}

\begin{signature} Let $F$ be a sequence of length $n$
such that $\Set{F}{1}{n} \subseteq \mathbb{N}$.
$\Prod{F}{1}{n}$ is a natural number.
\end{signature}

\begin{axiom}[Factorproperty]
  Let $F$ be a sequence of length $n$
  such that $F(i)$ is a nonzero natural number for every element $i$ of $\Dom{F}$.

  Then $\Prod{F}{1}{n}$ is nonzero and for every element $i$ of $\Dom{F}$
    there is a natural number $m$ such that $m = F(i)$ and $m$ divides $\Prod{F}{1}{n}$.
\end{axiom}


\end{forthel}

\section{Finite and Infinite Sets}

\begin{forthel}

\begin{definition} $S$ is finite iff 
$S = \Set{F}{1}{n}$ for some natural number $n$ and some function $F$ that is 
a sequence of length $n$.
\end{definition}

\begin{definition} $S$ is infinite iff $S$ is not finite.
\end{definition}

\end{forthel}

\section{Euclid's Theorem}

\begin{forthel}

\begin{signature}
  $\Primes$ is a class.
\end{signature}

\begin{axiom}
  $\Primes$ is the class of prime natural numbers.
\end{axiom}

\begin{theorem}[Euclid]
$\Primes$ is infinite.
\end{theorem}
\begin{proof}
Take a natural number $r$.
Take a function $p$ such that $p$ is a sequence of length $r$.
Assume that $\Set{p}{1}{r}$ is a subclass of $\Primes$.

%$\val{p}{i}$ is a nonzero natural number for every $i \in Dom(p)$.
(1) $\val{p}{i}$ is a nonzero natural number for every $i$ such
that $1 \leq i \leq r$.

Take a natural number $n$ such that $n=\Prod{p}{1}{r}+1$.
Take a prime divisor $q$ of $n$.

Let us show that $q \neq \val{p}{i}$ for all $i$ such that  $1 \leq i \leq r$.
  Assume the contrary.
  Assume that $q=\val{p}{i}$ for some natural number $i$ such that
  $1 \leq i \leq r$.
  $q$ is a divisor of $n$ and $q$ is a divisor of $\Prod{p}{1}{r}$
  (by Factorproperty, 1).
  Thus $q$ divides $1$. Contradiction.
qed.

Hence $\Set{p}{1}{r}$ is not the class of prime natural numbers.
\end{proof}
\end{forthel}

\end{document}