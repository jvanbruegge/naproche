\documentclass{article}

\title{The mutilated checkerboard}
\author{Adrian De Lon, Peter Koepke, Anton Lorenzen}
\date{2021}

\usepackage[utf8]{inputenc}
\usepackage[english]{babel}
\usepackage{../tex/naproche}
\usepackage{bbm}
\usepackage{hyperref}

% For the ranks and files
\newcommand{\Rank}{\mathbf{R}} % set of ranks.
\newcommand{\File}{\mathbf{F}} % set of files.
\newcommand{\fileA}{\mathrm{a}}
\newcommand{\fileB}{\mathrm{b}}
\newcommand{\fileC}{\mathrm{c}}
\newcommand{\fileD}{\mathrm{d}}
\newcommand{\fileE}{\mathrm{e}}
\newcommand{\fileF}{\mathrm{f}}
\newcommand{\fileG}{\mathrm{g}}
\newcommand{\fileH}{\mathrm{h}}

\newcommand{\vertadj}{\sim}
\newcommand{\horizadj}{\sim'} % should be \wr

\newcommand{\Checkerboard}{\mathbf{C}}
\newcommand{\Corners}{\mathbf{C'}}
\newcommand{\Mutilated}{\mathbf{M}}
\newcommand{\Black}{\mathbf{B}}
\newcommand{\White}{\mathbf{W}}

\newcommand{\Swap}[1]{\operatorname{Swap}#1}
\newcommand{\Sw}[3]{\operatorname{Swap}_{#1}^{#2}(#3)}

\newcommand{\Naproche}{$\mathbbm{N}$aproche}


\begin{document}

\maketitle





\section{Introduction}


\
Max Black proposed this problem in his book
\textit{Critical Thinking} (1946).
%
It was later discussed by Martin Gardner in his
\textit{Scientific American} column, \textit{Mathematical Games}.
%
John McCarthy, one of the
founders of Artificial Intelligence described it as a
\textit{Tough Nut for Proof Procedures}
and discussed fully automatic or interactive proofs of the solution.

There have been several formalization of the Checkerboard problem before. A
survey article by Manfred Kerber and Martin Pollet called \emph{A Tough Nut
for Mathematical Knowledge Management} lists a couple of formalizations.

\section{Setting up the checkerboard}

% Hidden because we have not discussed it yet.
% Maybe this can be mentioned as an area for improvement in the subsequent section?
\begin{comment}
\begin{forthel}
    [readtex prelude.ftl.tex]
    [synonym rank/ranks] [synonym file/files] [synonym square/squares] [synonym domino/dominoes]
\end{forthel}
\end{comment}

We introduce \textit{types} (or \textit{notions}) and constants
to model checkerboards as a Cartesian product of
\textit{ranks} $1, 2, \ldots, 8$ and \textit{files} $\fileA, \fileB, \ldots, \fileH$, where we follow standard checkerboard
notation.
In future versions of \Naproche{} these signature declarations 
should be grouped as a single declaration of an inductively defined set, 
allowing phrasings such as \textit{1, 2, 3, 4, 5, 6, 7, 8 are ranks}.
Note that the effect of signature declarations is to extend
the underlying first-order language. \Naproche{} treats $1, 2, \ldots$ as new constant symbols which have no connection
to the homonymous integers and in particular do not carry assumptions about distinctness. Here our approach diverges from
McCarthy's who employs integers modulo $8$, but this would require
us to formalize part of the theory of $\mathbb{Z}/8 \mathbb{Z}$.

\Naproche{} allows to group elements into \textit{classes} and \textit{sets} as long as they are \textit{setsized} (informally also called \textit{small}).

\begin{forthel}
    \begin{signature} A rank is an object. \end{signature}
    Let $r, s$ denote ranks.

    \begin{signature} $1$ is a rank. \end{signature}
    \begin{signature} $2$ is a rank. \end{signature}
    \begin{signature} $3$ is a rank. \end{signature}
    \begin{signature} $4$ is a rank. \end{signature}
    \begin{signature} $5$ is a rank. \end{signature}
    \begin{signature} $6$ is a rank. \end{signature}
    \begin{signature} $7$ is a rank. \end{signature}
    \begin{signature} $8$ is a rank. \end{signature}

    \begin{definition} $\Rank = \{1,2,3,4,5,6,7,8\}$. \end{definition}

    \begin{signature} A file is an object. \end{signature}
    Let $f, g$ denote files.

    \begin{signature} $\fileA$ is a file. \end{signature}
    \begin{signature} $\fileB$ is a file. \end{signature}
    \begin{signature} $\fileC$ is a file. \end{signature}
    \begin{signature} $\fileD$ is a file. \end{signature}
    \begin{signature} $\fileE$ is a file. \end{signature}
    \begin{signature} $\fileF$ is a file. \end{signature}
    \begin{signature} $\fileG$ is a file. \end{signature}
    \begin{signature} $\fileH$ is a file. \end{signature}

    \begin{definition}
        $\File = \{\fileA,\fileB,\fileC,\fileD,\fileE,\fileF,\fileG,\fileH\}$.
    \end{definition}

    \begin{signature} A square is an object. \end{signature}
    \begin{signature} $(f, r)$ is a square. \end{signature}
    Let $v, w, x, y, z$ denote squares.
\end{forthel}

Is there a set of all squares? This may not be true for an arbitrary notion,
but it is true for squares, so we assume it as an axiom. Note that we can always
form the class $C$ of all inhabitants of a notion as long as $x \in C$ can only be true
for setsized $x$.
Morse and Kelley \cite{Kelley1975,Morse1965} use the same approach in their axiomatization of set theory.

\begin{forthel}
    \begin{signature}
        $\Checkerboard$ is a set.
    \end{signature}

    \begin{axiom}
        $\Checkerboard$ = { square x | $x = (f, r)$ for some
            element $f$ of $\File$ and some element $r$ of $\Rank$ }.
    \end{axiom}
\end{forthel}

\section{Preliminaries about sets and functions}

We enrich the small built-in set theory with further properties and axioms that
will be used in the course of our argument. To keep the document fully
self-contained we formulate the necessary definitions and axioms ourselves.
Note that there are many degrees of freedom in picking an axiomatic setting.

In Definition 27, we do not define the \emph{relation}
$A \subseteq B$,
but rather the \emph{type} of subsets of a given set $B$.
The type is parametrized by a member of the type \textit{set}.
This is a kind of dependent type, depending on $B$, in the sense of dependent type theory.
%
Therefore the wording \textit{A \elision{} is a \elision{}} is used in
Definition 27 as well as in Definition 29.
%
More specifically, we use a form with an argument prefixed by \textit{of},
a so-called \textit{{of}}-notion.
This allows certain grammatical variants and constructs like \textit{$B$ has a subset $A$} or
\textit{every subset of $A$ satisfies \elision{}}.
%
Note that an \textit{element} similarly is an \textit{{of}}-notion so that one could
write phrases like \textit{$x$ is an element of $B$} or even
more complicated ones like
\textit{for all nonequal elements $A, B$ of $C$}.
%
One could also have defined the \emph{relation} $A \subseteq B$
by a statement of the form:
\textit{$A \subseteq B$ iff \elision{}}.
%
Definition 30 defines
disjointness in that format.

\begin{forthel}
    Let $A,B,C$ denote sets.

    \begin{definition}
        A subset of $B$ is a set $A$ such that
        every element of $A$ is an element of $B$.
    \end{definition}

    \begin{axiom}[Extensionality]
        If $A$ is a subset of $B$ and
        $B$ is a subset of $A$ then $A = B$.
    \end{axiom}

    \begin{definition}
        A proper subset of $B$ is a subset $A$ of $B$ such that $A\neq B$.
    \end{definition}

    \begin{definition}
        $A$ is disjoint from $B$ iff there is no element of $A$ that is an element of $B$.
    \end{definition}

    \begin{definition}
        A family is a set $F$ such that every element of $F$ is a set.
    \end{definition}

    \begin{definition}
        Let $F$ be a family.
        A member of $F$ is a set $A$ such that $A$ is an element of $F$.
    \end{definition}
    
    \begin{definition}
        A disjoint family is a family $F$ such that for all nonequal sets $A, B$
        such that $A, B$ are elements of $F$ $A$ is disjoint from $B$.
    \end{definition}

    \begin{definition}
        $B \cap C = \{ x "in" B \mid x \in C \}$.
    \end{definition}

    \begin{definition}
        $B \setminus C = \{ x "in" B \mid x \notin C \}$.
    \end{definition}
\end{forthel}

The notion of \textit{object}
is the built-in \textit{largest notion}, containing all other notions.
%
Also note that the proof of the lemma below really is omitted and not merely hidden:
with its internal ``reasoner'' and in non-trivial cases with the help of automated theorem provers
such as \textsf{E}, \Naproche{}
can accept some theorems without any additional argumentation.

The built-in ordered pair notation that we already used in the first
section does not include the universal property of ordered pairs,
so we postulate it as an axiom.

\begin{forthel}
    \begin{axiom}
        Let $\alpha, \beta, \gamma, \delta$ be objects.
        If $(\alpha, \beta) = (\gamma, \delta)$
        then $\alpha = \gamma$ and $\beta = \delta$.
    \end{axiom}
\end{forthel}

(Unary) functions are built into \Naproche{}; $F(t)$ denotes the
application of a function $F$ to an argument $t$ and
$\Dom{F}$ stands for the domain of $F$. In our exposition we shall
use functions to compare cardinalities of black and white squares. As with
sets, we introduce some further properties of functions.

\begin{forthel}
    Let $F,G$ denote functions.

    \begin{definition}
        $F : A \to B$ iff $\Dom{F} = A$ and
        $F(x)$ is an element of $B$ for all elements $x$ of $A$.
    \end{definition}
\end{forthel}

Bijective functions are the basis of the modern theory of cardinalities; sets
have the same cardinality iff there is a bijection between them.

\begin{forthel}
    % TODO remove parentheses of quantifiers.
    \begin{definition}
        $F : A \leftrightarrow B$ iff $F : A \to B$ and there exists $G$ such that
        $G : B \to A$ and
        (for all elements $x$ of $A$ we have $G(F(x)) = x$) and
        (for all elements $y$ of $B$ we have $F(G(y)) = y$).
    \end{definition}
\end{forthel}

\section{Cardinalities of Finite Sets}

\begin{forthel}
    \begin{definition}
        $A$ is equinumerous with $B$ iff there is $F$ such that $F : A \leftrightarrow B$.
    \end{definition}

    \begin{lemma}
        Assume that A is equinumerous with B. Then B is equinumerous with A.
    \end{lemma}

    % TODO prettify proof (try to allow for subscripts in variables?)
    % Use symbolic quantifiers to avoid uncanny valley of parenthesized natural language text?
    %
    \begin{lemma}
        Assume that A is equinumerous with B and B is equinumerous with C.
        Then A is equinumerous with C.
    \end{lemma}
    \begin{proof}
        Take a function $F$ such that $F : A \leftrightarrow B$.
        Take a function $G$ such that $G : B \to A$ and (for all elements $x$ of $A$ we have $G(F(x)) = x$) and
        (for all elements $y$ of $B$ we have $F(G(y))=y$).
        Take a function $H$ such that $H : B \leftrightarrow C$.
        Take a function $I$ such that $I : C \to B$ and (for all elements $x$ of $B$ we have $I(H(x)) = x$) and
        (for all elements $y$ of $C$ we have $H(I(y))=y$).
        Define $J' = \{ (x, H(F(x))) \mid x \in A \}$.
        Take a function $J$ such that $J = J'$.
        $J : A \leftrightarrow C$. Indeed define $K = \{ (y, G(I(y))) \mid y \in C \}$.
    \end{proof}
\end{forthel}

For the finite checkerboard problem we only need to consider finite sets.
We can thus assume that all sets considered are finite, and then
we have the following finiteness axiom:

\begin{forthel}
    \begin{axiom}
        If $A$ is a proper subset of $B$ then $A$ is not equinumerous with $B$.
    \end{axiom}
\end{forthel}

\section{The Mutilated Checkerboard}

Defining the mutilated checkerboard is straightforward:
we simply remove the two corners.

\begin{forthel}
    \begin{definition}
        $\Corners = \{ (\fileA, 1), (\fileH, 8) \}$.
    \end{definition}

    \begin{definition}
        $\Mutilated = \Checkerboard \setminus \Corners$.
    \end{definition}

    Let the mutilated checkerboard stand for $\Mutilated$.
\end{forthel}

\section{Dominoes}

To define dominoes,
we introduce concepts of adjacency by first declaring new relations and then axiomatizing them.
As usual, chaining of relation symbols indicates a conjunction.

\begin{forthel}
    \begin{signature}
        $r$ is vertically adjacent to $s$ is a relation.
    \end{signature}
    Let $r\vertadj s$ stand for $r$ is vertically adjacent to $s$.

    \begin{axiom}
        If $r\vertadj s$ then $s\vertadj r$.
    \end{axiom}
    \begin{axiom}
        $1\vertadj 2\vertadj 3\vertadj 4\vertadj 5\vertadj 6\vertadj 7\vertadj 8$.
    \end{axiom}

    \begin{signature}
        $f$ is horizontally adjacent to $g$ is a relation.
    \end{signature}
    Let $f\horizadj g$ stand for $f$ is horizontally adjacent to $g$.

    \begin{axiom}
        If $f \horizadj g$ then $g \horizadj f$.
    \end{axiom}
    \begin{axiom}
        $\fileA \horizadj \fileB \horizadj \fileC \horizadj \fileD
        \horizadj \fileE \horizadj \fileF \horizadj \fileG \horizadj \fileH$.
    \end{axiom}

    \begin{definition}
        $x$ is adjacent to $y$ iff
        there exist $f, r, g, s$ such that
        $x = (f, r)$ and $y = (g, s)$ and
        (($f = g$ and $r$ is vertically adjacent to $s$) or
         ($r = s$ and $f$ is horizontally adjacent to $g$)).
    \end{definition}

    \begin{definition}
        A domino is a set $D$ such that $D = \{x,y\}$ for some
        adjacent squares $x, y$.
    \end{definition}
\end{forthel}

\section{Domino Tilings}

\begin{forthel}
    \begin{definition}
        A domino tiling is a disjoint family $T$ such that every member of $T$ is a domino.
    \end{definition}

    Let $A$ denote a subset of $\Checkerboard$.

    \begin{definition}
        A domino tiling of $A$ is a domino tiling $T$
        such that for every square $x$
        $x$ is an element of $A$ iff
        $x$ is an element of some member of $T$.
    \end{definition}
\end{forthel}

\noindent
We shall prove:

\begin{quotedtheorem}
    The mutilated checkerboard has no domino tiling.
\end{quotedtheorem}



\section{Colours}

We shall solve the mutilated checkerboard problem by a cardinality argument.
Squares on an actual checkerboard are coloured black and white and we can count
colours on dominoes and on the mutilated checkerboard $\Mutilated$.

The introduction of colours can be viewed as a creative move typical of
mathematics: changing perspectives and introducing aspects that are not part
of the original problem. The mutilated checkerboard was first discussed under
a cognition-theoretic perspective: can one solve the problem \emph{without}
inventing new concepts and completely stay within the realm of squares,
subsets of the checkerboard and dominoes.


\begin{forthel}
    \begin{signature} $x$ is black is a relation. \end{signature}
    \begin{signature} $x$ is white is a relation. \end{signature}

    \begin{axiom} $x$ is black iff $x$ is not white. \end{axiom}
    \begin{axiom} If $x$ is adjacent to $y$ then $x$ is black iff $y$ is white. \end{axiom}

    \begin{axiom} $(\fileA,1)$ is black. \end{axiom}
    \begin{axiom} $(\fileH,8)$ is black. \end{axiom}

    \begin{definition} $\Black$ is $\{ x "in" \Checkerboard | "there is a square s such that s = x and s is black" \}$. \end{definition}
    \begin{definition} $\White$ is $\{ x "in" \Checkerboard | "there is a square s such that s = x and s is white" \}$. \end{definition}
\end{forthel}


\section{Counting Colours on Checkerboards}

The original checkerboard has an equal number of black and white squares.
Since our setup does not include numbers for counting, we rather work with
equinumerosity. The following argument formalizes that we can invert the
colours of a checkerboard by swapping the files $\fileA$ and $\fileB$, $\fileC$ and $\fileD$, and so on.
We formalize swapping by a first-order function symbol $\Swap{}$.

\begin{forthel}
    \begin{signature}
        $\Swap{x}$ is a square.
    \end{signature}

    \begin{axiom}
        Let $x$ be a square.
        Assume $x$ is an element of $\Checkerboard$.
        $\Swap{x}$ is an element of $\Checkerboard$.
    \end{axiom}

    Let $t$ denote a rank.

    \begin{axiom} $\Swap{(\fileA, t)} = (\fileB, t)$ and $\Swap{(\fileB, t)} = (\fileA, t)$. \end{axiom}

    \begin{axiom} $\Swap{(\fileC, t)} = (\fileD, t)$ and $\Swap{(\fileD, t)} = (\fileC, t)$. \end{axiom}

    \begin{axiom} $\Swap{(\fileE, t)} = (\fileF, t)$ and $\Swap{(\fileF, t)} = (\fileE, t)$. \end{axiom}

    \begin{axiom} $\Swap{(\fileG, t)} = (\fileH, t)$ and $\Swap{(\fileH, t)} = (\fileG, t)$. \end{axiom}
\end{forthel}

The somewhat unsightly case-splits in the following lemmas
are necessary to guide the prover,
since \Naproche{} as yet has no concept of finite data types and
only features well-ordered induction.
As a consolation price, we can omit the last case.

\begin{forthel}
    \begin{lemma}
        Let $x$ be a square.
        Assume $x$ is an element of $\Checkerboard$.
        $\Swap{x}$ is adjacent to $x$.
    \end{lemma}
    \begin{proof}
        Take $f$, $r$ such that $x = (f, r)$.
        $r$ is an element of $\Rank$.
        Case $f = \fileA$. End.
        Case $f = \fileB$. End.
        Case $f = \fileC$. End.
        Case $f = \fileD$. End.
        Case $f = \fileE$. End.
        Case $f = \fileF$. End.
        Case $f = \fileG$. End.
    \end{proof}
\end{forthel}

$\Swap{}$ is an involution.
% TODO

\begin{forthel}
    \begin{lemma}
        Let $x$ be a square.
        Assume $x$ is an element of $\Checkerboard$.
        $\Swap{(\Swap{x})} = x$.
    \end{lemma}
    \begin{proof}
        Take $f, r$ such that $x = (f,r)$. $r$ is an element of $\Rank$.
        Case $f = \fileA$. End.
        Case $f = \fileB$. End.
        Case $f = \fileC$. End.
        Case $f = \fileD$. End.
        Case $f = \fileE$. End.
        Case $f = \fileF$. End.
        Case $f = \fileG$. End.
    \end{proof}

    \begin{lemma}
        Let $x$ be a square.
        Assume $x$ is an element of $\Checkerboard$.
        $x$ is black iff $\Swap{x}$ is white.
    \end{lemma}
\end{forthel}

%From Swap we can define a map which testifies that:
Using $\Swap$ we can define a witness of $\Black \leftrightarrow \White$.

\begin{forthel}
    \begin{lemma}
        $\Black$ is equinumerous with $\White$.
    \end{lemma}
    \begin{proof}
        Define $F' = \{ (x, \Swap{x}) | "x is a square and" x \in \Black \}$.
        Define $G' = \{ (x, \Swap{x}) | "x is a square and" x \in \White \}$.
        Take a function $F$ such that $F = F'$.
        Take a function $G$ such that $G = G'$.
        Then $F : \Black \to \White$ and $G : \White \to \Black$.
        For all elements $x$ of $\Black$ we have $G(F(x)) = x$.
        For all elements $x$ of $\White$ we have $F(G(x)) = x$.
        $F : \Black \leftrightarrow \White$.
    \end{proof}
\end{forthel}

Given a domino tiling one can also swap the squares of each domino,
leading to similar properties.

\begin{forthel}
    \begin{signature}
        Assume that $T$ is a domino tiling of $A$.
        Let $x$ be an element of $A$.
        $\Sw{T}{A}{x}$ is a square $y$ such that there is an element $D$ of $T$
        such that $D = \{x,y\}$.
    \end{signature}

    \begin{lemma}
        Assume that $T$ is a domino tiling of $A$.
        Let $x$ be an element of $A$.
        Then $\Sw{T}{A}{x}$ is an element of $A$.
    \end{lemma}
    \begin{proof}
        Let $y = \Sw{T}{A}{x}$.
        Take an element $D$ of $T$ such that $D = \{x,y\}$.
    \end{proof}
\end{forthel}


Swapping dominoes is also an involution.

\begin{forthel}
    \begin{lemma}
        Assume that $T$ is a domino tiling of $A$. Let $x$ be an element of $A$.
        Then $\Sw{T}{A}{\Sw{T}{A}{x}} = x$.
    \end{lemma}
    \begin{proof}
        Let $y = \Sw{T}{A}{x}$.
        Take an element $Y$ of $T$ such that $Y = \{x,y\}$.
        Let $z = \Sw{T}{A}{y}$.
        Take an element $Z$ of $T$ such that $Z = \{y,z\}$.
        Then $x = z$.
    \end{proof}

    \begin{lemma}
        Assume that $T$ is a domino tiling of $A$.
        Let $x$ be a black element of $A$.
        Then $\Sw{T}{A}{x}$ is white.
    \end{lemma}
    \begin{proof}
        Let $y = \Sw{T}{A}{x}$.
        Take an element $Y$ of $T$ such that $Y = \{x,y\}$.
    \end{proof}
\end{forthel}

\section{The Theorem}

\noindent We can easily show that a domino tiling involves as many black as white squares.

\begin{forthel}
    \begin{lemma}
        Let $T$ be a domino tiling of $A$. Then $A \cap \Black$ is
        equinumerous with $A \cap \White$.
    \end{lemma}
    \begin{proof}
        Define $F = \{ (x, \Sw{T}{A}{x}) \mid x \in A \cap \Black \}$.
        Define $G = \{ (x, \Sw{T}{A}{x}) \mid x \in A \cap \White \}$.
        Then $F: A \cap \Black \to A \cap \White$ and
        $G: A \cap \White \to A \cap \Black$.
        For all elements $x$ of $A \cap \Black$ we have $G(F(x))=x$.
        For all elements $x$ of $A \cap \White$ we have $F(G(x))=x$.
        $F : A \cap \Black \leftrightarrow A \cap \White$.
    \end{proof}
\end{forthel}

\noindent In mutilating the checkerboard, one only removes black squares

\begin{forthel}
    \begin{lemma}
        $\Mutilated \cap \White = \White$.
    \end{lemma}
    \begin{proof}
        $\Mutilated \cap \White$ is a subset of $\White$.
        $\White$ is a subset of $\Mutilated$.
        Proof.
            Let $x$ be an element of $\White$.
            $x \neq (\fileA, 1)$ and $x \neq (\fileH,8)$.
            Indeed $(\fileH, 8)$ is black.
        End.
    \end{proof}
\end{forthel}

\noindent Now the theorem follows by putting together the previous cardinality properties.
Note that the phrasing \textit{[...]\,has no domino tiling} in the theorem is automatically derived from the definition of \textit{a domino tiling of\,[...]}.

\begin{forthel}
    \begin{theorem}
        The mutilated checkerboard has no domino tiling.
    \end{theorem}
    \begin{proof}
        Proof by contradiction.
        Assume $T$ is a domino tiling of $\Mutilated$.
        $\Mutilated \cap \Black$ is equinumerous with $\Mutilated \cap \White$.
        Indeed $\Mutilated$ is a subset of $\Checkerboard$.
        $\Mutilated \cap \Black$ is equinumerous with $\White$.
        $\Mutilated \cap \Black$ is equinumerous with $\Black$.
        Contradiction. Indeed $\Mutilated \cap \Black$ is a proper subset of $\Black$.
    \end{proof}
\end{forthel}

\begin{thebibliography}{1}

\bibitem{eprover} Stephan Schulz et al., \textit{E}, \url{https://eprover.org}

\bibitem{Kelley1975} John L. Kelley, \textit{General Topology}, Springer, 1975

\bibitem{Morse1965} Anthony Perry Morse, Trevor J McMinn, \textit{A theory of sets}, Academic press, 1965

\end{thebibliography}

\end{document}
